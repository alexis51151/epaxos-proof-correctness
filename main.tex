%
% This example is based on:
% http://www.maths.tcd.ie/~dwilkins/LaTeXPrimer/Theorems.html
%
\documentclass[a4paper]{article}

\usepackage[english]{babel}
\usepackage[utf8]{inputenc}
\usepackage{amsmath,amsthm,amssymb}

% Creating cases lists up to a depth of 3
\usepackage{enumitem}
\newlist{case}{enumerate}{3}
\setlist[case,1]{label=Case \arabic*., align=left, leftmargin=*, labelindent=0cm}
\setlist[case,2]{label*=\roman*., align=left, leftmargin=0cm}
\setlist[case,3]{label*=\alph*., align=left, leftmargin=0cm}




\renewcommand\qedsymbol{$\blacksquare$}

\title{Egalitarian Paxos: Proof of correctness}
\author{Alexis Le Glaunec}

\theoremstyle{definition}
\newtheorem{definition}{Definition}

\theoremstyle{plain}
\newtheorem{theorem}{Theorem}
\newtheorem{lemma}{Lemma}
\newtheorem{property}{Property}
\newtheorem{invariant}{Invariant}



      
\begin{document}
\maketitle




\begin{definition}[Pre-Accept]
\begin{align*}
preaccept(D,c,Q,b) \triangleq \forall p \in Q, \lozenge( & deps(c)=D \\
                                                         & \wedge vbal=b=0 \\
                                                         & \wedge status(c) = "preaccepted")
\end{align*}
\end{definition}

\begin{definition}[Accept]
\begin{align*}
    accept(D,c,Q,b) \triangleq \forall p \in Q, \lozenge( & deps(c)=D \\
                                                             &\wedge vbal=b \\
                                                             &\wedge status(c) = "accepted")
\end{align*}
\end{definition}

\begin{definition}[Vote]
\begin{align*}
    vote(D,c,p,b) \triangleq  \lozenge( & p.deps(c)=D \\
                                                             &\wedge p.vbal=b > 0 \\
                                                             &\wedge p.status(c) = "accepted")
\end{align*}
\end{definition}


\begin{definition}[Committable]
\begin{align*}
    committable(D,c,Q,b) \triangleq  preaccept(D,c,Q,b) \vee accept(D,c,Q,b)
\end{align*}
\end{definition}

\begin{definition}[Committed]
\begin{align*}
    committed(D,c) \triangleq  \exists p \in Replicas, \lozenge(& deps(c) = D \\
                                                                &\wedge status(c) = "committed") 
\end{align*}

\end{definition}

\begin{definition}[Executed]
\begin{align*}
    executed(D,c) \triangleq  \exists p \in Replicas, \lozenge(& deps(c) = D \\
                                                                &\wedge status(c) = "executed") 
\end{align*}

\end{definition}





\begin{property}
$committed(D,c) \implies \exists \; b,Q  \; committable(D,c,Q,b)$
\end{property}

\begin{proof}
Let's suppose $committed(D,c)$. We consider $cleader \in Replicas$, the set of replicas, the leader of the command first to have $deps(c) = D \wedge status(c) = "committed"$. Therefore $committed(D,c) \implies \lozenge Phase1Fast(cleader,i,Q) \vee \lozenge Phase2Finalize(cleader,i,Q)$.

\begin{case}
    \item $\lozenge Phase1Fast(cleader,i,Q)$\\
    $\lozenge Phase1Fast(cleader,i,Q) \implies \lozenge StartPhase1(c,cleader,Q,i,0,oldMsg)$ and a $StartPhase1$ postcondition is $vbal=b$. Thus we have $vbal=b=0$. Moreover one of $Phase1Fast$ preconditions is $\forall p \in Q, (p.deps(c) = cleader.deps(c))$. As $\lozenge Phase1Fast \implies \forall p \in q, p.status(c) = "preaccepted"$, we conclude that $preaccept(D,c,Q,b)$ and therefore $committable(D,c,Q,b)$.
    
    \item $\lozenge Phase2Finalize(cleader,i,Q)$\\
    This is a similar case, replacing $p.status="preaccepted"$ with $p.status="accepted"$ and $\lozenge Phase2Finalize \implies \lozenge Phase1Slow(cleader,i,Q)$. Therefore as $vbal=b$ is a postcondition of $Phase1Slow$, we conclude that $accept(D,c,Q,b)$ and therefore $committable(D,c,Q,b)$.
\end{case}
\end{proof}

\begin{property}
\begin{equation*}
    vote(D,c,p,b) \implies \forall b'<b, \; (committable(D',c,Q',b') \implies D=D')
\end{equation*}
\end{property}

\begin{proof}
By induction on $b$, the ballot number. \\
    \underline{Induction hypothesis:}\\
$vote(D,c,p,b) \implies \forall b'<b, \; (committable(D',c,Q',b') \implies D=D')
$\\

By definition, the base case is true. Let's consider a ballot $b>0$ as we suppose $vote(D,c,p,b)$.  At a recovery step, only $\lozenge PrepareFinalize$ leads to $vote$. Let's suppose $\exists b_M<b$ the highest ballot with $\; committable(D_M,c,Q_M,b_M)$ and by induction $\forall b'<b_M, \; committable(D',c,Q',b') \implies D=D'$. Replica $p$ is part of quorum $Q$. Let's define $R=Q \cap Q_M$ and by definition $R \neq \emptyset$ hence $\exists r \in R$. If $vote(D_r,c,r,b) < vote(D'_r,c,r,b_M)$, it is in contradiction with the ballot order $b < b_M$ that is a precondition of $PrepareFinalize$ therefore impossible. Thus $vote(D_r,c,r,b) > vote(D'_r,c,r,b_M)$, and by definition of $b_M$, $\nexists q, vote(D_q,c,q, b_q)$. Therefore we take $D=D_M$ as $b_M$ is the highest ballot lower than $b$ possibly $committable$, and the induction hypothesis is verified.
\end{proof}


\begin{property}
$committable(D,c,Q,b) \wedge committable(D',c,Q',b') \implies D=D'$
\end{property}
\begin{proof}
We suppose $committable(D,c,Q,b) \wedge committable(D',c,Q',b')$.
\begin{case}
    \item $b = b'$ 
    \begin{case}
        \item $b=0$\\
        At ballot $b=0$, $\exists! cleader, \lozenge Propose(c, cleader)$ as $Propose$ precondition $c \notin proposed$ is completed with $Propose$ postcondition $proposed' = proposed \cup \{c\}$. Thus $cleader$ proposes only once a set of dependences $D$ at a quorum $Q$. Therefore, there is only $committable(D,c,Q,0)$ with $D$ and $Q$ uniques.
        \item $b>0$ \\
        There can be several leaders recovering a command at the same time. However, as ballots are totally ordered by lexicographical order on $(ballot, replica)$ and that EPaxos is a majority-based protocol, only one $cleader$ has committable(D,c,Q,b). And as for the previous case,  $SendPrepare$ preconditions guarantee that $cleader$ will propose a unique set of dependencies $D$ to a quorum $Q$.
    \end{case}
    \item $b > b'$ \\
    By induction on $b$, the ballot number. \\
    \underline{Induction hypothesis:}
    \[committable(D,c,Q,b) \implies (\forall b' < b, committable(D',c,Q',b') \implies D' = D)\]\\
    By definition of the induction hypothesis, $b>0$ and the base case is true. In the recovery step, $committable$ is accessible only through $\lozenge PrepareFinalize$. We define $replies$ the set of replies from a quorum $Q$ to the new leader $cleader$, and consider the different cases regarding $replies$ content.
    
    \begin{case}
        \item  $\exists \; com \in replies \; with \; com.status \in \{"committed", "executed"\}$
        As $executed \implies committed$, we conclude by property 1 that $committable(D_M,c,Q_M,b_M)$ happened and take $D = D_M$. Therefore by induction comes the result.
        
        \item $\exists \; acc \in replies \; with \; acc.status = "accepted"$ \\
        Since we have $acc \in replies$, it means that $\exists p \in Q,  \exists  b_M < b,  vote(D_M,c,p,b_M).$ \\
        By property 2, we have $\forall b''<b_M, \; (committable(D'',c,Q'',b'') \implies D''=D_M).$ Hence by induction, as we choose $D=D_M=D''$, it verifies the induction hypothesis.
        
        % TODO: following case to check : looks suspicious to me
        \item $\forall msg \in replies, msg.status \notin \{"accepted", "committed", "executed"\}$ \\
        Then if $committable(D',c,Q',b')$, necessarily b'=0 by definition of \textit{pre-accept}. Let's consider $R = Q \cap Q'$. By definition of a quorum, $R \neq \emptyset$.
        
        \begin{case}
            \item $\forall p,q \in R, \; p.deps(c) = q.deps(c)=D'$ \\
            Therefore we choose $D=D'$.        
            
            \item $\exists p,q \in R, \; p.deps(c) \neq q.deps(c)$ \\
            It is in contradiction with $committable(D',c,Q',b')$. Hence, such $b'$ doest not exist and there is no constraint on $D$. \\
        \end{case}
    \end{case}

\end{case}
Therefore the induction hypothesis is verified in any case, hence comes the result.
\end{proof}


\begin{invariant}
\begin{equation*}
    committed(D,c) \wedge committed(D',c) \implies D=D'
\end{equation*}
\end{invariant}

\begin{proof}
Direct by combining properties 1 and 3.
\end{proof}

\begin{definition}[Sent]
\begin{align*}
     \exists m \in Sent \Longleftrightarrow  \lozenge(\exists m \in sentMsg)
\end{align*}
\end{definition}



\begin{definition}[Seen]
\begin{align*}
    seen(D,c,b,p) \triangleq \lozenge(& \exists m \in Sent, \; m.type \in \{"preaccept", "preaccept-reply", "try-preaccept-reply"\} \\
                                    &\wedge m.src = p \\
                                    &\wedge m.cmd = c \\
                                    &\wedge (m.type \neq try-preaccept-reply \implies m.deps = D) \\
                                    &\wedge (m.type = preaccept \implies b = 0 \vee m.ballot = b))
\end{align*}
\end{definition}

\begin{property}
\begin{equation*}
    vote(D,c,p,b) \implies \exists Q', \; \forall p \in Q', \; seen(D_p,c,b,p) \wedge (D = \bigcup_{p \in Q'}{D_p})
\end{equation*}
\end{property}

\begin{proof}
\end{proof}



\begin{property}
    \begin{align*}
    commitable(D,c,Q,b) \implies \exists Q', \; \forall p \in Q', \; seen(D_p,c,p,b) \wedge (D = \bigcup_{p \in Q'}{D_p})
    \end{align*}
\end{property}

\begin{proof}
By induction on b, the ballot number.

\begin{flushleft}
\underline{Induction hypothesis:}\\
\end{flushleft}
$ commitable(D,c,Q,b) \implies \exists Q', \; \forall p \in Q', \; seen(D_p,c,p,b) \wedge (D = \bigcup_{p \in Q'}{D_p})$\\


\begin{flushleft}
\underline{Base case:} $b = 0$
\end{flushleft}

\begin{case}
    \item $preaccept(D,c,Q,0)$ \\
    Let's consider the initial leader of the command $cleader$ and \\ $\lozenge StartPhase1(c,cleader,Q,i,b,\{\})$. The message is different depending on the nature of $p \in Q$:
    \begin{case}
        \item $p=cleader$\\
        Therefore $p$ sends a message $m$ with $m.src=p,m.cmd=c,m.deps= \{rec.inst : rec \in cmdLog[cleader]\}$ and $m.type = preaccept$.
        \item $p \neq cleader$\\
        Therefore $p$ replies to $m$ with $m_r$ having $m_r.src=p,m_r.cmd=c,m_r.deps= m.deps \cup ({t.inst : t \in cmdLog[p]} \backslash \{m.inst\})$ and $m_r.type = preaccept-reply$.
    \end{case}
    Hence, $\exists Q, \; \forall p \in Q, \; seen(D_p,c,p,b) \wedge (D = \bigcup_{p \in Q}{D_p})$
    
    \item $accept(D,c,Q,0)$ \\
    Therefore $\lozenge Phase1Slow(cleader,i,Q)$ and as a consequence, like for the previous case, $cleader$ sent a message in $\lozenge StartPhase1$ and $\forall p \in Q, p \neq cleader$, $p$ replied in $\lozenge Phase1Reply$. Thus $cleader$ sends a message $m$ with $m.deps = \cup \{m_r.deps : m_r \in replies\}$ where replies is the union of all $m_r$ from the previous case. Hence, by going through the two calls developed in the case above, $seen$ condtion is checked. And as $cleader$ proposes $D=m.deps$, we have $D = \bigcup_{p \in Q'}{D_p}$.
\end{case}
    \begin{flushleft}
    \underline{Induction step:} $b > 0$
    \end{flushleft}
    For \textit{c} to be \textit{committable}, there must be $\lozenge PrepareFinalize(replica, i, Q)$.\\
    
    \begin{case}
        \item $\exists \; com \in replies \; with \; com.status \in \{"committed", "executed"\}$ \\
        Therefore, $\exists b' < b , committable(D',c,Q',b')$. Taking $D=D'$, by induction, we have that $\exists Q', \; \forall p \in Q', \; seen(D_p,c,p,b) \wedge (D = \bigcup_{p \in Q'}{D_p})$.\\

        \item $\exists \; acc \in replies \; with \; acc.status = "accepted" $ \\
        By definition, we have $vote(D,c,p,b)$. Hence by property 4, $\exists Q', \; \forall p \in Q', \; seen(D_p,c,p,b) \wedge (D = \bigcup_{p \in Q'}{D_p})$.

        \item $\forall msg \in replies, msg.status \notin \{"accepted", "committed", "executed"\}$ \\
        Let's define $preaccepts \triangleq \{msg \in replies : msg.status = "preaccepted" \}$.
        
        \begin{case}
            \item $(|preaccepts| \geq |Q|-1)  \wedge (\forall m_1,m_2 \in preaccepts, m_1.deps = m_2.deps) \wedge (\forall m \in preaccepts : m.src \neq i[1])$\\
            Hence we have $|Q|-1$ replicas $p$ with $seen(D,c,b',p)$ without the leader, and $b'=0$ by definition of \textit{pre-accept}. As the initial leader picked the set of dependencies, it also had $seen(D,c,b',cleader)$. Therefore $\forall p \in Q, seen(D,c,b',p)$ and the induction hypothesis is true.
            
            \item $(|Q|-1 > |preaccepts| \geq |Q|/2  )  \wedge (\forall m_1,m_2 \in preaccepts, m_1.deps = m_2.deps) \wedge (\forall m \in preaccepts : m.src \neq i[1])$ \\
             $committable(D,c,Q,b) \implies \lozenge FinalizeTryPreAccept(cleader,i,Q)$. Let's define $tprs \triangleq \{msg \in sentMsg : msg.type = "try-preaccept-reply" \wedge msg.dst = cleader \wedge msg.inst = i \wedge msg.ballot = rec.ballot\}$. To be committable, we have either: 
            \begin{itemize}
                \item $\forall tpr \in tprs : tpr.status = "OK"$ which means that $\forall p \in Q, seen(D,c,p,b)$ and $D$ is chosen to be committable.
                \item $\exists tpr \in tprs : tpr.status \in \{"accepted", "committed", "executed"\}$, hence we initiate $StartPhase1$. We deal with this case in the following case.
            \end{itemize}

            \item Remaining cases \\
            In these cases, $committable \implies \lozenge StartPhase1$. It can be reduced to the base case with $Phase1Slow$ (we can only call \textit{StartPhase1} on a slow quorum at a ballot $b > 0$).
            As the base case verifies the induction hypothesis, this case verifies it too.


        \end{case}

    \end{case}
Therefore the induction hypothesis is verified in any case, hence comes the result.

\end{proof}

\begin{property}
\begin{equation*}
    \square(seen(\_,c,p,\_)) \implies \square(c \in cmdLog[p])
\end{equation*}
\end{property}

\begin{proof}
For the 3 sorts of message type, the message is sent after modifying $cmdLog[p]$ accordingly to the message content, hence the result.
\end{proof}


\begin{property}
\begin{equation*}
    \square(seen(D,c,p,b)) \implies \square(p.deps[c] \subseteq D)
\end{equation*}
\end{property}

\begin{proof}
\end{proof}

\begin{property}
\begin{equation*}
    seen(D,c,p,b) \wedge seen(D',c',p,b') \implies c \in D' \vee c' \in D
\end{equation*}
\end{property}

\begin{proof}
Let's suppose, without loss of generality, $seen(D,c,p,b) < seen(D',c',p,b')$.
By property 6, let's consider a time when $\square(c \in cmdLog[p])$. Then let's consider the first time when $\square seen(D',c',p,b')$. It can happen in 3 different cases according to the type of message:

\begin{case}
    \item $m.type = "preaccept"$ \\
    It means that $p$ is the leader of the command. As $c \in cmdLog[p]$, necessarily $c \in D'$ because $p$ sends a message $m$ with $m.deps = \{rec.inst : rec \in cmdLog[p]\}$.  

    \item $m.type = "preaccept-reply"$ \\
    It means that $p$ is a follower. As $c \in cmdLog[p]$, necessarily $c \in D'$ because $p$ receives a message $msg$ and reply with a message $m$ with $m.deps = msg.deps \cup \{t.inst : t \in cmdLog[p]\}\backslash \{msg.inst\}$.

    \item $m.type = "try-preaccept-reply"$ \\
    If $c \in D'$, the set proposed by the leader of the recovery, then the result is verified. Otherwise, if $c \notin D'$ then $c' \in p.deps[c]$ and as $seen(D,c,p,b)$, by property 8 we conclude that $p.deps[c] \subseteq D$ hence $c' \in D$.
\end{case}

\end{proof}

\begin{invariant}
    \begin{align*}
    committed(D,c) \wedge committed(D',c') \implies c \in D' \; or \; c' \in D
    \end{align*}
\end{invariant}

\begin{proof}
Direct by combining properties 1, 5 and 8.
\end{proof}



\end{document}